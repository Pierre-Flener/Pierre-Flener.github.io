\documentclass[a4paper,11pt,hidelinks]{article}
\usepackage[utf8]{inputenc}  % Linux, macOS: enable non-English characters
%\usepackage[latin1]{inputenc}    % Windows: enable non-English characters
\usepackage[left=2.5cm,right=2.5cm,top=2.5cm,bottom=2.5cm]{geometry}
\usepackage[british]{babel}

%% Instructions are preceded by "%%", whether you use LaTeX or not!

%\renewcommand{\thepart}{\arabic{part}}
\renewcommand{\thesection}{\Alph{section}}

%% Useful packages:
\usepackage[boxed]{algorithm}   % drop [boxed] is no box around algorithm wanted
\usepackage[noend]{algorithmic} % drop [noend] if endif, endwhile, etc wanted
\renewcommand{\algorithmiccomment}[1]{\hfill // #1}
\usepackage{alltt}
\usepackage{amsmath}
\usepackage{amssymb}
\usepackage{array}
\usepackage[british]{babel}
\usepackage{booktabs}
\usepackage{cancel}  % provides \cancel
\usepackage{xcolor}  % supersedes color
\usepackage{colortbl}  % provides \cellcolor
\usepackage[nocenter]{cwpuzzle}  % provides Sudoku
\usepackage{graphicx}  % provides \includegraphics
\usepackage{logicpuzzle}  % provides kakuro
\definecolor{kakuro}{RGB}{155,206,167}
\kakurosetup{color=kakuro}
\usepackage{mathtools} \mathtoolsset{showonlyrefs}  % provides \coloneqq
\usepackage{multicol}
\usepackage{multirow}
\usepackage{pdflscape}  % provides "landscape" environment
\usepackage{pgfplots} \pgfplotsset{compat=1.12}  % was 1.9, but gave errors?!
\usepackage{rotating}  % provides "sideways[table]" environments
\usepackage{tikz}
\usetikzlibrary{arrows,automata,calc,trees,positioning,decorations.markings}
\usepackage{xspace}
\usepackage{xstring}
\usepackage{hyperref}
\hypersetup{linkcolor=black}  % should always be last package

%% Fancy characters:
\usepackage{pifont}
%\newcommand{\scissors}{\ding{36}}
%\newcommand{\phone}{\ding{37}}
%\newcommand{\aircraft}{\ding{40}}
%\newcommand{\envelope}{\ding{41}}
\newcommand{\handpoint}{\ding{43}}
%\newcommand{\victory}{\ding{44}}
%\newcommand{\handwrite}{\ding{45}}
\newcommand{\tick}{\ding{51}}
\newcommand{\notick}{\ding{55}}
\DeclareSymbolFont{extraup}{U}{zavm}{m}{n}
%\DeclareMathSymbol{\varclubsuit}{\mathalpha}{extraup}{84}
%\DeclareMathSymbol{\varspadesuit}{\mathalpha}{extraup}{85}
\DeclareMathSymbol{\varheartsuit}{\mathalpha}{extraup}{86}
\DeclareMathSymbol{\vardiamondsuit}{\mathalpha}{extraup}{87}

%% Styles:
\newcommand{\important}[1]{\textcolor{red}{#1}}
\newcommand{\stressed}[1]{\textbf{\textit{#1}}}
\newcommand{\todo}[1]{\textcolor{green}{#1}}  % show to-do items in green
%\newcommand{\todo}[1]{}                      % omit to-do items
\newcommand{\done}[1]{\textcolor{blue}{#1}}   % show done  items in blue
%\newcommand{\done}[1]{#1}                    % show done  items in black

%% Tools and Languages:
\newcommand{\FlatZinc}{Flat\-Zinc\xspace}
\newcommand{\Gecode}{Gecode\xspace}
%\newcommand{\GIST}{GIST\xspace}
\newcommand{\MiniCP}{Mini\-CP\xspace}
\newcommand{\MiniModel}{Mini\-Model\xspace}
\newcommand{\MiniZinc}{Mini\-Zinc\xspace}
\newcommand{\Python}{Python\xspace}

%% Algorithms:
\newcommand{\EndIf}{\textbf{~endif}}
\newcommand{\Function}{\textbf{function~}}
\newcommand{\IfThen}[2]{\If#1\Then#2\EndIf}  % use when \IF unsuitable
\newcommand{\In}{\textbf{~in~}}
\newcommand{\Invariant}[1]{{\bf invariant:} #1}
%\newcommand{\IsAssigned}{\gets}
\newcommand{\IsAssigned}{\coloneqq}
\newcommand{\Let}[2]{\textbf{let~}#1\In#2}
\newcommand{\PostCond}[1]{\textbf{post:} #1}
\newcommand{\PreCond}[1]{\textbf{pre:} #1}
\newcommand{\Return}{\textbf{return~}}  % use when \RETURN unsuitable
\newcommand{\Select}{\textbf{select}}
\newcommand{\Variant}[1]{{\bf variant:} #1}

%% Automata:
\newcommand{\Alphabet}{\Sigma}
\newcommand{\Alternation}[2]{#1|#2}           % of regular expressions
\newcommand{\Concat}[2]{#1 \cdot #2}          % of strings
\newcommand{\DFA}{\mathcal{A}}
\newcommand{\EmptyString}{\epsilon}
\newcommand{\Kleene}[1]{#1^*}                 % of a regular expression
\newcommand{\Language}[1]{\mathcal{L}({#1})}  % of a regular expression
\newcommand{\RegEx}[1]{\mathbf{#1}}           % transforms symbol into reg.ex.

%% CP & Gecode & MiniCP & MiniZinc:
\newcommand{\AllDifferent}{\Constraint{AllDifferent}}  % MiniCP
\newcommand{\AnyCond}[1]{\text{Any}(#1)}  % Gecode
\newcommand{\BinPacking}{\Constraint{binpacking}}  % Gecode
\newcommand{\BoundedCond}[1]{\text{Bounded}(#1)}  % Gecode
\newcommand{\Channel}{\Constraint{channel}}  % Gecode
%\newcommand{\Circuit}{\Constraint{circuit}}  % Gecode
\newcommand{\Circuit}{\Constraint{Circuit}}  % MiniCP
\newcommand{\CondSet}[1]{\text{PropConds}(#1)}  % Gecode
\newcommand{\Conds}[2]{\text{Conds}(#1,#2)}  % Gecode
%\newcommand{\Constraint}[1]{\textsc{#1}}
\newcommand{\Constraint}[1]{\texttt{#1}}
%\newcommand{\Cumulative}{\Constraint{cumulative}}  % Gecode
\newcommand{\Cumulative}{\Constraint{Cumulative}}  % MiniCP
\newcommand{\DepProps}{\textit{DepProps}}  % Gecode
\newcommand{\DFE}{\text{DFE}}  % Gecode
\newcommand{\Disjunctive}{\Constraint{Disjunctive}}  % MiniCP
\newcommand{\Distinct}{\Constraint{distinct}}  % Gecode
\newcommand{\Domain}[1]{\textnormal{dom}(#1)}
%\newcommand{\Element}{\Constraint{element}}  % Gecode
\newcommand{\Element}{\Constraint{Element}}  % MiniCP
\newcommand{\Extensional}{\Constraint{extensional}}  % Gecode
\newcommand{\Failed}{\text{Failed}}  % Gecode
\newcommand{\FailedCond}[1]{\text{Failed}(#1)}  % Gecode
\newcommand{\FixedCond}[1]{\text{Fixed}(#1)}  % Gecode
\newcommand{\Fixpoint}{\text{AtFixpt}}  % Gecode
\newcommand{\GlobalCardinality}{\Constraint{count}}  % Gecode
\newcommand{\Linear}{\Constraint{linear}}  % Gecode
\newcommand{\MaxCond}[1]{\text{Max}(#1)}  % Gecode
\newcommand{\MinCond}[1]{\text{Min}(#1)}  % Gecode
\newcommand{\ModVars}{\textit{ModVars}}  % Gecode
\newcommand{\NoneCond}[1]{\text{None}(#1)}  % Gecode
\newcommand{\NValues}{\Constraint{nvalues}}  % Gecode
\newcommand{\Path}{\Constraint{path}}  % Gecode
\newcommand{\Precede}{\Constraint{precede}}  % Gecode
\newcommand{\Propagate}{\text{Propagate}}
%\newcommand{\Reifies}[2]{#1\Leftrightarrow#2}
\newcommand{\Reifies}[2]{#2\Leftrightarrow#1}  % layout as in MiniZinc
\newcommand{\Stores}{\mathbb{S}}
\newcommand{\Subsumed}{\text{Subsumed}}  % Gecode
\newcommand{\Sum}{\Constraint{Sum}}  % MiniCP
\newcommand{\Table}{\Constraint{Table}}  % MiniCP
\newcommand{\Unary}{\Constraint{unary}}  % Gecode
\newcommand{\Unknown}{\text{Unknown}}  % Gecode
\newcommand{\Variables}[1]{\text{var}(#1)}

%% Mathematics:
\newcommand{\AbsValue}[1]{\left\lvert#1\right\rvert}
\newcommand{\Cardinality}[1]{\left\lvert#1\right\rvert}
%\newcommand{\Cardinality}[1]{\##1}
\newcommand{\Ceiling}[1]{\left\lceil#1\right\rceil}
\newcommand{\Else}{\textbf{~else~}}
\newcommand{\EmptySet}{\varnothing}
\newcommand{\Floor}[1]{\left\lfloor#1\right\rfloor}
\newcommand{\GeqLex}{\geq_{\Lex}}
\newcommand{\If}{\textbf{if~}}
\newcommand{\Iff}{\Leftrightarrow}
\newcommand{\IfThenElse}[3]{\If#1\Then#2\Else#3}
\newcommand{\Implies}{\Rightarrow}
\newcommand{\Int}{\mathbb{Z}}
\newcommand{\Inter}{\cap}
\newcommand{\Iverson}[1]{\red{\left[\textcolor{black}{#1}\right]}}
\newcommand{\LeqLex}{\leq_{\Lex}}
\newcommand{\Lex}{\textnormal{lex}}
\newcommand{\LtLex}{<_{\Lex}}
\newcommand{\Nat}{\mathbb{N}}
\newcommand{\Oh}[1]{\mathcal{O}(#1)}
\newcommand{\Sequence}[1]{\left[#1\right]}
\newcommand{\Set}[1]{\left\{#1\right\}}
\newcommand{\SetComp}[2]{\Set{#1\SuchThat#2}}
\newcommand{\SuchThat}{\mid}
\newcommand{\Then}{\textbf{~then~}}
\newcommand{\Tuple}[1]{\left\langle#1\right\rangle}
\newcommand{\Union}{\cup}
\newcommand{\Where}{\textbf{~~where~~}}

\usepackage{listings}
\usepackage{courier} % \texttt{...} gives thinner text and /\ displays OK

\newcommand\mznfont{\fontfamily{pcr}\selectfont}

\lstdefinelanguage{Mzn}
{
  morekeywords={
  %
  array, par, var, opt, constraint, solve, satisfy, minimize,
  maximize, output, include, let, in, set, of, if, then, else, elseif, endif,
  ann, annotation, bool, enum, float, int, string, where, function,
  predicate, true, false, not, assert, trace,
  % ???:
  any, list, op, record, test, tuple, type,
  %
  },
  %
  keywords=[2]{
  %
  forall, exists, xor, xorall, iffall, clause,
  all_different, all_different_int,
  all_different_except_0, all_different_except, all_equal,
  nvalue, diffn,
  at_least, at_most, exactly, % deprecated!
  count, count_eq, count_leq, count_geq, count_gt, among,
  global_cardinality, global_cardinality_closed,
  global_cardinality_low_up, global_cardinality_low_up_closed,
  element, regular, regular_nfa, table, inverse, inverse_in_range,
  bin_packing, bin_packing_capa, bin_packing_load, knapsack,
  cumulative, disjunctive, circuit, subcircuit, dpath,
  decreasing, increasing,
  strictly_decreasing, strictly_increasing,
  lex_less, lex_lesseq, lex_greater, lex_greatereq, lex2, strict_lex2,
  value_precede, value_precede_chain,
  symmetry_breaking_constraint, implied_constraint, redundant_constraint,
  sort, arg_sort, among, sliding_sum,
  int_ne, int_lt_reif, int_lin_eq, int_lin_eq_reif, bool_lin_eq, int_eq_imp,
  bool_imply, array_bool_or,
  in_set, subset, superset, partition_set, member,
  % from ???:
  abort,
  acosh, asin, atan, cos, cosh, sin, sinh, tan, tanh,
  array_intersect, array_union,
  array1d, array2d, array3d, array4d, array5d, array6d,
  bool2int, int2float, set2array,
  abs, div, mod, pow, exp, sqrt, ln, log, log2, log10,
  ceil, floor, round,
  enum_next,
  min, max, length, product, sum,
  dom, dom_array, dom_size, fix, is_fixed,
  index_set, index_set_1of2, index_set_2of2,
  index_set_1of3, index_set_2of3, index_set_3of3, index_set_6of6,
  concat, reverse, join,
  lb, lb_array, ub, ub_array,
  show, show2d, show_int, show_float,
  card, intersect, union, diff, symdiff,
  % annotations:
  add_to_output,
  is_defined_var, output_var, var_is_introduced, defines_var, promise_total,
  value_propagation, bounds_propagation, domain_propagation,
  bool_search, int_search, seq_search,
  set_search, input_order, first_fail, anti_first_fail, smallest,
  largest, occurrence, most_constrained, max_regret, indomain_min,
  indomain_max, indomain_middle, indomain_median, indomain,
  indomain_random, indomain_split, indomain_reverse_split,
  indomain_interval, outdomain_max, outdomain_median, outdomain_min,
  outdomain_random, complete, restart_constant, relax_and_reconstruct
  % 
  },
  sensitive=true,
  basicstyle=\mznfont,
  commentstyle=\color[rgb]{0.9,0.1,0.1},
  keywordstyle=\color[rgb]{0,0.5,0},
  keywordstyle=[2]\color{blue},
  stringstyle=\color{orange},
  tabsize=2,
  frame=none,
  % identifierstyle = \it,
  numbers=left,
  stepnumber=1,
  numberstyle=\tiny,
  numbersep=5pt,
  xleftmargin=0pt, % numbers will be in the margins!
  columns=fixed, % same width for all characters
  % columns=flexible,
  % columns=fullflexible,
  morecomment=[l]{\%},
  morestring=[b]",
  % morestring=[d]',
  showstringspaces=false,
  mathescape=true,
  breaklines=true,
  % prebreak=\raisebox{0ex}[0ex][0ex]{\ensuremath{\space\red{\swarrow}}},%\hookrightarrow
  %postbreak=\raisebox{0ex}[0ex][0ex]{\ensuremath{\hookrightarrow\space}},%\hookrightarrow
  breakatwhitespace=true,
  breakindent=10pt, % was: 20pt
  moredelim=**[is][\color{Melon}]{@}{@},
  escapeinside={{<@}{@>}}
}
%% Write "\begin{frame}[fragile]" for a slide using either of the
%% following two listing environments, which have unnumbered
%% respectively numbered lines:
\lstnewenvironment{mzn}[1][]{\lstset{language=Mzn,#1}}{}
\lstnewenvironment{mznno}[1][]{\lstset{language=Mzn,numbers=none,xleftmargin=0pt,#1}}{}
%% Inline a code snippet, without respectively with the comprehension bar (|):
\newcommand{\mzninline}[1]{\lstinline[{language=Mzn}]|#1|}
\newcommand{\mzninlinebar}[1]{\lstinline[{language=Mzn}]!#1!}


\renewcommand{\todo}[1]{\textcolor{blue}{#1}} %% Use for spotting placeholders
%\renewcommand{\todo}[1]{#1}                  %% Use for the submitted report

\newcommand{\Problem}{\todo{Problem}\xspace}  %% Plug in your project name

\title{\textbf{Modelling for Combinatorial Optimisation (1DL451) \\
    Uppsala University -- Autumn~2025 \\
    Report for the Project
    by Team~\todo{t}: \\                      %% Replace t by your team number
    \Problem
  }
}

%% Replace by your name(s) and choose the encoding of line 2 or line 3:
\author{\todo{Clara CLÄVER and Whiz KIDD}}

\date{\today}

\begin{document}

\maketitle

% -------------------------------------------------------------------------

\section{\Problem}

%% Describe your project problem, in your own words, citing the source
%% of the problem and of every included third-party picture.

\todo{\dots\ Reply \dots}

% -------------------------------------------------------------------------

\section{Approach}

%% Describe your approach to your project problem.
%% If your approach is just a single model (like for the assignments),
%% then just say so and follow the model instructions of demoReport.pdf,
%% else describe your software pipeline -- with pre-processing,
%% solving (possibly on a sequence of models), and post-processing --
%% and follow the model instructions of demoReport.pdf for _each_ model.
%% Either way, upload _all_ code and data to Studium (_except_ for the
%% initial report), but _only_ import your model(s) into the report itself).
%% The report must be about exactly one approach (and this word is _not_
%% a synonym for 'viewpoint') and must _not_ describe all false starts.

\todo{\dots\ Reply \dots}

%% Find in demoReport.pdf the instructions on prescribed comments within a model.
Our model is given in Listing~\ref{model:project}: it has the
prescribed comments as per the scope (Topics~1 to~8) of the project,
it has the name \todo{\texttt{project.mzn}}             %% Replace the filename
and the imposed structure of the provided skeleton model
\texttt{project-skeleton.mzn}%
%% Comment away the following line for the initial report
, and it is uploaded to Studium.

%% Set the numeric parameters so that the copyright notice is skipped:
\lstinputlisting[language=Mzn, firstline=5, firstnumber=5,
caption={A \MiniZinc model for \todo{\dots} },
label=model:project]{project-skeleton.mzn} %% Replace the filename

\paragraph{Symmetries.}
%% * Identify the problem symmetries, which exist for every viewpoint.
%% * Identify the model symmetries, which are introduced by your viewpoint.
%% * For each identified symmetry, state:
%%   + whether it is a value symmetry or a variable symmetry
%%     (an example of which is an index symmetry, say a row or column symmetry);
%%   + whether it is a full or partial symmetry.
\todo{\dots\ problem symmetry \dots\ model symmetry \dots\ value /
  variable / index / row / column symmetry \dots\ full / partial
  symmetry \dots}

\paragraph{Efficiency.}
%% * State the impact, or justify the absence, of
%%   + implied constraints, and
%%   + symmetry-breaking constraints.
%%   Do _not_ test them on CBLS backends, such as Yuck, which ignore them.
%% * State the impact, or justify the absence, of
%%   + reasoning annotations, and
%%   + search annotations.
%%   Restrict your tests to the CP backend Gecode.
\todo{\dots\ implied constraints \dots\ symmetry-breaking constraints
  \dots\ reasoning annotations \dots\ search annotations \dots}

\paragraph{Checklist.}
%% For each model feature covered by advice in the checklists of Topics 2 & 3:
%% how do you argue that it does not matter?
\todo{\dots\ Reply (and we understand that we may lose points if there
  are such model features that we did not detect and discuss) \dots}

\paragraph{Correctness.}
%% How do you argue that your approach is correct?
%% For example, do you use a checker based on another model, such as a
%% model by another team or a linearisation of a MIP model found somewhere?
%% Do you use the same new instances as another team?
%% If it is a satisfaction problem: do you compare the numbers of
%% solutions with numbers found somewhere or reported by another team?
%% If it is an optimisation problem: do you compare objective values that
%% are proven optimal before timing out (respectively objective values
%% known when timing out) with optimal objective values (respectively
%% best-known objective values) found somewhere or reported by another
%% team?)
\todo{\dots\ Reply \dots}

% -------------------------------------------------------------------------

\section{Evaluation}

%% Hint: Under Linux, do lscpu to find CPU information.  Under macOS,
%% you find CPU information via "About This Mac" in the Apple menu.
%% 
%% If different hardware was used for different experiments, then justify
%% this and replicate a paragraph like this one within each relevant section.

All experiments were run under Linux Ubuntu~22.04.5 ($64$~bit) on an
Intel Xeon E5520 of $2.27$~GHz, with $4$~processors of $4$~cores each,
with a $70$~GiB RAM and an $8$~MiB L3 cache (a ThinLinc computer of the
IT department). % barany.it.uu.se

\newcommand{\TimeOut}{\todo{6,000,000}} % in CPU milliseconds  %% Choose a value

%% You _must_ use the script of our cheatsheet: it conducts the
%% experiments and generates a result table (see the LaTeX source code
%% of the table below}) that is automatically imported (rather than
%% manually copied) into your report, so each time you change the
%% model, it suffices to re-run that script and re-compile your
%% report, without any tedious number copying!

Table~\ref{tab:res} gives our results.
The time-out was~\TimeOut~milliseconds.

%\begin{landscape}  %% Uncomment this if the table is too wide!
\begin{table}[t]
  \centering
  \resizebox{\columnwidth}{!}{\fontfamily{pcr}\selectfont\small
    \begin{tabular}{rrrrrrrrrrrrr}
      Backend
	& \multicolumn{2}{c}{Chuffed}
	& \multicolumn{2}{c}{CP-SAT}
	& \multicolumn{2}{c}{Gecode}
	& \multicolumn{2}{c}{Gurobi}
	& \multicolumn{2}{c}{PicatSAT}
	& \multicolumn{2}{c}{Yuck}
\\
	\cmidrule(lr){2-3}
	\cmidrule(lr){4-5}
	\cmidrule(lr){6-7}
	\cmidrule(lr){8-9}
	\cmidrule(lr){10-11}
	\cmidrule(lr){12-13}
        \dots
	& \texttt{obj} & time
	& \texttt{obj} & time
	& \texttt{obj} & time
	& \texttt{obj} & time
	& \texttt{obj} & time
	& \texttt{obj} & time
\\
        \midrule
        \dots
	& \dots	& \dots
	& \dots	& \dots
	& \dots	& \dots
	& \dots	& \dots
	& \dots	& \dots
	& \dots	& \dots
\\
  %% Replace the filename
    \end{tabular}}
  \caption{Results for our approach to \Problem,
    which is a \todo{minimisation / maximisation / satisfaction} problem.
    % 
    In each \texttt{time} column: if the reported time is less than
    the time-out (\TimeOut~milliseconds here), then the
    %% For a CSP, tweak the following snippet as in Table 1 for Assignment 1:
    \todo{objective value in the corresponding \texttt{obj} column was
      \emph{proven} optimal};
    % 
    else the timing out is indicated by \texttt{t/o} and the
    %% For a CSP, tweak the following snippet as in Table 1 for Assignment 1:
    \todo{objective value is either the best one found but \emph{not}
      proven optimal before timing out,
      or~`\texttt{-}' indicating that no feasible solution was found
      before timing out}.
    % 
    Boldface indicates the
    %% For a CSP, replace by "fastest time":
    \todo{best performance (time or objective value)}
    %
    on each row.
    % 
  }
  \label{tab:res}
\end{table}
%\end{landscape}

\paragraph{Which backends win overall, and how do you draw that conclusion?}
\todo{\dots\ Reply \dots}

\paragraph{How do the backends scale, and how do you draw that conclusion?}
\todo{\dots\ Reply \dots}

\paragraph{Does the difficulty of instances monotonically increase
  with their size, and how do you draw that conclusion?}
\todo{\dots\ Reply \dots}

\paragraph{How suitable is local search compared to systematic search,
  and how do you draw that conclusion?}
\todo{\dots\ Reply \dots}

\paragraph{Are there any contradictions between the results?}
\todo{\dots\ Reply \dots}

\paragraph{Are there any occurrences of `ERR' within the results generated
  by the experiment script?}
%% If so, then first try and troubleshoot on your own by running the
%% incriminated backend manually (within the IDE or at the command
%% line by using the --solver flag of the minizinc command) and
%% interpreting the error message.  If you cannot resolve the error,
%% then you _must_ state here for each occurrence of ERR when and how
%% you received a teacher's _prior_ approval to include it, and you
%% ought to make an error report in the final section.
\todo{\dots\ Reply \dots}

% -------------------------------------------------------------------------

%% Optional:

\bigskip
\section*{Feedback to the Teachers}

%% Please write a paragraph, which will _not_ be graded, describing your
%% experience with this report: which aspects were too difficult or
%% too easy, and which aspects were interesting or boring?  This will
%% help us improve the course for the next year.

\todo{\dots\ Reply \dots}

% -------------------------------------------------------------------------

%% Optional:

\section*{Error Report}

%% Your models _must_ compile and run error-free under backends of
%% _all_ the considered solving technologies, unless you have a
%% teacher’s _prior_ approval to upload an error report here.

\todo{\dots\ Reply \dots}

% -------------------------------------------------------------------------

%% Optional:

% \bibliographystyle{plain}
% \bibliography{M4CO}

\end{document}

%%% Local Variables:
%%% mode: latex
%%% TeX-master: t
%%% End:
