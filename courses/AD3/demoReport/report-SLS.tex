\section*{Problem 2: Stochastic Local Search (SLS)}

\todo{
  %% delete this paragraph, including its footnote:
  The \emph{investment design problem} is about finding a matrix of
  $v$~rows and $b$~columns of~$0$-$1$ integer values, such that each
  row sums up to~$r$, with $v \geq 2$ and $b \geq r \geq 1$, and the
  largest dot product between all pairs of rows is minimised.
  % 
  Equivalently, one has to find~$v$ subsets of size~$r$ within a given
  set of $b$~elements, such that the largest intersection of any two
  of the $v$~sets has minimal size.
  % 
  An instance of the problem is parametrised by a
  triple~$\Tuple{v,b,r}$.\footnote{\todo{Your report need not contain
      an explanation of the problem to be solved: you can assume the
      reader has read the problem statement in the assignment
      instructions.  We just repeat it here for self-sufficiency of
      this document.}}
  
  %% delete this paragraph:
  This is an abstract description of a problem that appears in
  finance: see Figure~\ref{fig:wallStreet}.  In a typical investment
  design in finance, we have $4 \leq v \leq 25$ and
  $250 \leq b \leq 500$, with $r \approx 100$.

  %% delete this figure:
  \begin{figure}[b]  % make it float to the bottom of a page
    \centering
    \includegraphics[height=5cm]{wallStreet}
    \caption{\todo{Wall Street (\copyright\ www.forbes.com, 2014)}}
    \label{fig:wallStreet}
  \end{figure}
  
  %% delete this paragraph, if you want:
  A lower bound on the number~$\lambda$ of shared elements of any
  pair among $v$~subsets of size~$r$ drawn from a given set of
  $b$~elements is given in~\cite{ASTRA:AOC}:
  \begin{equation}\label{eq:lb}
  \text{lb}(\lambda) = 
    \Ceiling{
      \frac{\Ceiling{\frac{rv}{b}}^{2} ((rv) \bmod b)  +
        \Floor{\frac{rv}{b}}^{2} (b - ((rv) \bmod b)) - rv}
      {v(v-1)}
    }
  \end{equation}
  
}

\paragraph{Task~a: SLS Algorithm.}
\begin{enumerate}
\item Representation.  \todo{Describe how to represent the problem:
  what are the variables, their meanings, their constraints, and the
  objective function?}
\item Initial Assignment.  \todo{Describe an algorithm for generating
  (fast) a randomised initial assignment.}
\item Move.  \todo{Describe one or more moves that go from an
  assignment to a neighbouring assignment by changing the values of
  a few variables.}
\item Constraints.  \todo{Describe for each constraint how its
  satisfaction is either algorithmically checkable efficiently or
  guaranteed to be preserved by the previous two design choices.}
\item Neighbourhood.  \todo{Describe a neighbourhood based on the
  proposed moves.
  % 
  Derive a formula for computing the size of the neighbourhood in
  terms of the problem parameters.
  % 
  Discuss whether the neighbourhood makes the search space connected,
  in the sense that every feasible assignment (that is, every
  assignment satisfying all the constraints, whether optimal or not)
  is reachable from every initial assignment (you only need to sketch
  a proof if the search space is connected, and give a counterexample
  otherwise).}
\item Cost Function.  \todo{Describe a cost function, whose value is
  to be minimised during search.}
\item Probing.  \todo{Describe how a neighbouring assignment, as
  reachable by a move, can be probed efficiently:
  % 
  describe how the cost function can be evaluated efficiently and
  incrementally, and describe the data structures used to do so.
  % 
  Give, without proof, the time complexity of probing; ideally, it is
  (sub-)linear in the problem parameters.}
\item Heuristic.  \todo{Describe a heuristic for exploring (via
    probing) the neighbourhood and selecting a neighbouring assignment
    to commit to.
  % 
  State whether the neighbourhood is explored exhaustively and, if
  so, how you determine when it was exhausted.
  % 
  Explain how you ensure that the same neighbour is not probed twice
  during a given exploration.}
\item Optimality.  \todo{Describe how you use a bound on the objective
  value in order to terminate sometimes the search with proven
  optimality, as part of the heuristic.}
\item Meta-Heuristic.  \todo{Describe a meta-heuristic based on tabu
  search: explain how the tabu list is represented; choose (a
  formula for) its size; explain how fine-grained its content is;
  and describe how it can be looked up and maintained efficiently;
  note that the tabu list is not necessarily an actual list, but
  rather a concept; make sure that worsening moves are sometimes
  made.}
\item Random Restarts.  \todo{Describe how to detect or guess that a
  random restart should be made, as part of the meta-heuristic.}
\item Optional Tweaks.  \todo{Describe ideas that you used in order to
  improve your algorithm.}
\end{enumerate}
In summary, the local-search parameters (not the problem
parameters~$v$,~$b$,~$r$) are~\todo{$\alpha$ and~$\beta$.}

\paragraph{Task~b: Implementation.}
We chose the high-performance programming language \todo{Java}, for
which a \todo{compiler or interpreter} is available on the Linux
computers of the IT department.  All source code is \todo{uploaded}
with this report (but not listed inside it).  The compilation and
running instructions are \todo{\filler}.

An executable called \texttt{InvDes} reads the problem
parameters~$v$,~$b$,~$r$ as command-line arguments and writes to
standard output a line with the space-separated values
of~$v$,~$b$,~$r$, the lower bound~$\text{lb}(\lambda)$ on~$\lambda$,
and the achieved~$\lambda$, followed by one line per row of
a~$v \times b$ matrix representing the solution, the~$0$-$1$ cell
values being space-separated.

We validated the correctness of our implementation by \todo{checking
  its outputs on 1,258 instances via the provided polynomial-time
  solution checker}.

\paragraph{Task~c: Experiments.}
%% Note the non-breaking spaces, typeset via ~, between numbers and
%% their units:
All experiments were run under
% specification of the ThinLinc machines of the IT department, or
% replace by a similar-looking specification of your own hardware:
\todo{Linux Ubuntu~18.04 ($64$~bit) on an Intel Xeon E5520 of
  $2.27$~GHz, with $4$~processors of $4$~cores each, with a $70$~GB
  RAM and an~$8$~MB L3 cache (a ThinLinc computer of the IT
  department).}

We fine-tuned the local-search parameters as follows.  \todo{Discuss
  the impact of the local-search parameters~$\alpha$ and~$\beta$ on
  the performance of your SLS algorithm.}

\newcommand{\TimeoutSLS}{\todo{300.0}}  % timeout, in CPU seconds; MIN 300.0
\newcommand{\RunsSLS}{\todo{5}}         % independent runs per instance; MIN 5

The \todo{median} runtime (in seconds), \todo{median} number of steps,
and \todo{median} achieved~$\lambda$ over~$\RunsSLS$ independent runs
for each of the $21$~instances of the assignment instructions are
given in Table~\ref{tab:res:sls}, for \todo{two} configurations
%% remember: AT LEAST TWO, except for solo teams!
of values for the local-search parameters~\todo{$\alpha$ and~$\beta$}.
The timeout was~$\TimeoutSLS$~CPU seconds per run.%
%
%% Delete this footnote:
\footnote{\todo{Hint: In order to save a lot of time, it is very
    important that you write a script that conducts the experiments
    for you and directly generates a result table (see the \LaTeX\
    source code of Table~\ref{tab:res:sls} for how to do that), which
    is then automatically imported, rather than manually copied, into
    your report: each time you change the code, it suffices to re-run
    that script and re-compile your report, without any tedious
    copying!  The sharing of scripts is allowed and even encouraged.}}

\begin{table}[t]  % make it float to the top of a page
  \centering
  \begin{tabular}{rrrrrrrrrrr}  % right alignment --> decimal point alignment
    & & &
    & \multicolumn{3}{c}{\todo{$\Tuple{\alpha,\beta}=\Tuple{10,5}$}}
    & \multicolumn{3}{c}{\todo{$\Tuple{\alpha,\beta}=\Tuple{20,8}$}} \\
    \cmidrule(r){5-7} \cmidrule(r){8-10}
    $v$ & $b$ & $r$ & $\text{lb}(\lambda)$
        & time & steps & $\lambda$
        & time & steps & $\lambda$ & exact \\
    \midrule
    % Make sure every number in a column has the _same_ number of decimals,
% so as to get decimal-point alignment and easy comparison of numbers!
%
% Witness in particular the 300.0 instead of 300!
%
10 &  30 &   9 &  2 & 300.0 &  123 & 3 &  45.0 &  678 & 2 & $10^{22}$ \\
12 &  44 &  11 &  2 & 300.0 &  901 & 3 & 234.5 & 5678 & 2 & $10^{21}$ \\
15 &  21 &   7 &  2 & 300.0 & 1023 & 4 & 300.0 & 6789 & 3 &      n/a \\
16 &  16 &   6 &  2 & 123.4 &  567 & 2 &  89.1 & 2345 & 2 & $10^{14}$ \\
 9 &  36 &  12 &  3 \\
11 &  22 &  10 &  4 \\
19 &  19 &   9 &  4 \\
10 &  37 &  14 &  5 \\
 8 &  28 &  14 &  6 \\
10 & 100 &  30 &  7 \\
 6 &  50 &  25 & 10 \\
 6 &  60 &  30 & 12 \\
11 & 150 &  50 & 14 \\
 9 &  70 &  35 & 16 \\
10 & 350 & 100 & 22 \\
13 & 250 &  80 & 22 \\
10 & 325 & 100 & 24 \\
15 & 350 & 100 & 24 \\
 9 & 300 & 100 & 25 \\
12 & 200 &  75 & 25 \\
10 & 360 & 120 & 32 \\
 %% let your experiment script write directly
                            %% into this file, making sure every number
                            %% in a column has the _same_ number of decimals
  \end{tabular}
  \caption{Investment design: median runtime (in seconds), median
    number of steps, and median achieved~$\lambda$, for \todo{two}
    configurations
    %% remember: AT LEAST TWO, except for solo teams!
    of values for the local-search parameters~\todo{$\alpha$
      and~$\beta$}, over~$\RunsSLS$~independent runs per instance,
    with a timeout of $\TimeoutSLS$~CPU seconds per run.
    %
    The right-most column gives the number of candidate solutions the
    outlined exact algorithm has to examine per second in order
    to match the runtime performance of the seemingly best
    configuration of values for the local-search parameters, namely
    \todo{$\Tuple{\alpha,\beta}=\Tuple{20,8}$}, if the instance
    was solved to proven optimality, and `n/a' for `non-applicable'
    otherwise.
    %% delete the following sentence:
    \todo{(The sample performance of this demo report is made up!)}
    %
  }
  \label{tab:res:sls}
\end{table}

We observe that \todo{\filler}, because \todo{\filler}.

\paragraph{Task~d: Exact Algorithm.}
An exact algorithm could work as follows: \todo{discuss its features
  (for instance, does it perform brute-force search?).}
%
The size of the search space of this exact algorithm is
\todo{$\binom{r!}{\cos b} \cdot \log v$}, because \todo{\filler}.

The number of candidate solutions this exact algorithm has to examine
per second in order to match the runtime performance of the seemingly
best configuration of values for the local-search parameters,
according to Task~c, of our stochastic local search algorithm is given
in the right-most column of Table~\ref{tab:res:sls}, for each instance
solved to proven optimality.
%
We think that \todo{\filler}, because \todo{\filler}.

%%% Local Variables:
%%% mode: latex
%%% TeX-master: "demoReport"
%%% End:
