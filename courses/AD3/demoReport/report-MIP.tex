\section*{Problem 1: Mixed Integer Programming (MIP)}

%% Fix below and drop the invocation of the \todo command:
\newcommand{\SolverMIP}{\todo{Gurobi}\xspace}  % or CPLEX, or Xpress, or Cbc
\newcommand{\TimeoutMIP}{\todo{300.00}}  % timeout, in CPU seconds; MIN 300.00

\paragraph{Task~a: Model.}~
\todo{What are the variables, their meanings, their constraints, and
  the objective function?
  % 
  For example, for the investment design problem, one might write:
  ``Let variable~$m_{ij}$ take value~$1$ if basket~$i$ invests in
  credit~$j$, and value~$0$ otherwise, with $i \in 1 \twodots v$ and
  $j \in 1 \twodots b$.
  % 
  The constraint that each row must sum up to~$r$ can then be linearly
  modelled as
  \[
    \forall i \in 1 \twodots v : \sum_{j \in 1 \twodots b} m_{ij} = r
  \]
  % 
  Etc.''
  %
}

\paragraph{Task~b: Implementation.}
Our model \texttt{servStatLoc.mod} is \todo{uploaded} with this report
(but not listed inside it): we \todo{checked} that its constraints and
objective function are linear (and we are aware that four points will
otherwise be deducted from our score for this problem).
%
We chose the MIP solver~\SolverMIP for our experiments, which we ran
\todo{on the NEOS server or
  % specification of the ThinLinc Linux hosts of the IT department,
  % where the academic site license for Gurobi is installed, but note
  % that no classroom license for AMPL is installed there, so replace
  % if need be by a similar-looking specification of your own hardware:
  under Linux Ubuntu~18.04 ($64$~bit) on an Intel Xeon E5520 of
  $2.27$~GHz, with $4$~processors of $4$~cores each, with a $70$~GB
  RAM and an $8$~MB L3 cache (a ThinLinc computer of the IT
  department)}.

\paragraph{Task~c: 10 Zones.}
The results are in Table~\ref{tab:res:mip}.
%% not needed from vt24 on:
% When~$s$ increases, the optimal objective value~\todo{\filler}.

\paragraph{Task~d: 20 Zones.}
The results are in Table~\ref{tab:res:mip}.
%% not needed from vt24 on:
% When~$s$ grows beyond~$4$, the optimal objective value \todo{\filler}.

\paragraph{Task~e: 40 Zones.}
The results are in Table~\ref{tab:res:mip}.

\paragraph{Task~f: 80 Zones.}
The results are in Table~\ref{tab:res:mip}.
%% not needed from vt24 on:
% Upon~$s=16$ service stations with~$v=1$ vehicle each, the optimal
% objective value is \todo{\filler} the one for~$s=8$ and~$v=2$, because
% \todo{\filler}.

\paragraph{Task~g: 120 Zones.}
The results are in Table~\ref{tab:res:mip}.
%
Our model \todo{does not time out}.
%% If it does time out, then see the demo sentence under Task h below!

\paragraph{Task~h: 250 Zones.}
The results are in Table~\ref{tab:res:mip}.
%
Our model \todo{times out, so our proposed algorithm for delivering a
  not necessarily optimal solution in reasonable running time is
  \filler}.

\paragraph{Task~i: Brute-Force Algorithm.}
The size of the search space of a totally brute-force search algorithm
is \todo{$\binom{z!}{\cos c} \cdot \log_s v$}, because \todo{\filler}.

The numbers of candidate solutions this brute-force search algorithm
has to examine per second in order to match the reported runtime
performance of~\SolverMIP on our model are given in the right-most
column of Table~\ref{tab:res:mip}, for each instance that~\SolverMIP
solved to proven optimality without timing out.
%
%% not needed from vt24 on:
% We think that \todo{\filler}, because \todo{\filler}.

\begin{table}[t]  % make it float to the top of a page
  \centering
  \begin{tabular}{rrrrrrrr}  % right alignment --> decimal point alignment
    $z$ & $s$ & $v$ & $c$ & time & objective value & optimality gap & brute-force \\
    \midrule
    % Make sure every number in a column has the _same_ number of decimals,
% so as to get decimal-point alignment and easy comparison of numbers!
%
% Witness in particular the 0.023246261350 instead of 0.02324626135 in line 4!
%
 10 &  2 & 2 & 3 &  \todo{67.89} & 0.008740338682 & 0.00\% & \todo{$10^{17}$} \\
 10 &  3 & 2 & 3 \\
 10 &  4 & 2 & 3 \\
 20 &  2 & 2 & 3 & \todo{123.45} & 0.023246261350 & 0.00\% & \todo{$10^{23}$} \\
 20 &  3 & 2 & 3 \\
 20 &  4 & 2 & 3 \\
 20 &  5 & 2 & 3 \\
 20 &  6 & 2 & 3 \\
 40 &  5 & 2 & 3 \\
 80 &  8 & 2 & 3 \\
 80 & 16 & 1 & 3 \\
120 & 10 & 2 & 3 \\
250 & 12 & 3 & 4 & \TimeoutMIP & & \todo{2.34\%} & \todo{n/a} \\
 %% let your experiment script write directly
                            %% into this file, making sure every number
                            %% in a column has the _same_ number of decimals
  \end{tabular}
  \caption{Service station location: runtime (in seconds), objective
    value, and optimality gap (in percent; positive if an optimal
    solution was not found and proven before timing out)
    using~\SolverMIP, with a timeout of $\TimeoutMIP$~CPU seconds.
    The right-most column gives the number of candidate solutions the
    brute-force search algorithm has to examine per second in order to
    match the runtime performance of~\SolverMIP, if the instance was
    solved to proven optimality, and~`n/a' for `non-applicable'
    otherwise.
    %% delete the following sentence:
    \todo{(The sample performance of this demo report is made up,
      but the two optimal objective values are correct!)}
    % 
  }
  \label{tab:res:mip}
\end{table}

%%% Local Variables:
%%% mode: latex
%%% TeX-master: "demoReport"
%%% End:
